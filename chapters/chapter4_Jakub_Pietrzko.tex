\section {Jakub Pietrzko} 
\label{sec:JakubP}
            pic rel 
\begin{figure}[htbp]  
    \centering
    \includegraphics[width = 0.5\textwidth]{pictures/buty_pana_wojtka.JPG}
    \caption{Nowe buty pana Wojtka}
    \label{fig:buty}
\end{figure}



Zima idzie, a ja w starych adidasach już miałem dziurę taką,
 że można było 2 palce wsadzić więc po miesiącu wyrzeczeń dzięki którym zaoszczędziłem pieniądze, 
poszedłem wczoraj do galerii handlowej kupić sobie porządne buty na zimę. W sklepie CCC znalazłem pic rel. 
Solidne wykonanie, przystępna cena, modny wygląd- nie zastanawiałem się długo. Wróciłem z butami do domu, pochodziłem w nich po pokoju,
 poprzeglądałem się w lustrze i czułem dobrze. Jeszcze je solidnie zaimpregnowałem, żeby nie przepuszczały wody i się nie niszczyły. 



Dzisiaj poszedłem w swoich nowych butach na AGH. Czułem się dzięki nim bardziej pewny siebie,
 jak siedziałem na korytarzu to nogi wyciągałem daleko, żeby ludzie lepiej widzieli jakie mam eleganckie buty. 
Po zajęciach czekam na przystanku na na autobus, a tu z kawiarni wychodzi znany podróżnik katolicki \textbf{Wojciech Cejrowski}.
 Elegancko ubrany, a nie w jakąś tam koszulę hawajską, z egzotycznych motywów to miał tylko w ręce taki kubeczek na yerba mate. 
Popatrzył się na mnie, na moje buty i podchodzi i zagaduje, czy te buty to są te z CCC za 139,00zł. 
Ja mu zadowolony mówię, że tak panie Wojtku, te same dokładnie, i że miło, że pan zauważył. 
Cejrowski na to powiedział tylko \underline{"co za slaby kolega"}



~\ref{tab:tabela_mocy} \begin{table}[]
\begin{tabular}{@{}lllll@{}}
\toprule
\textbf{1} & \multicolumn{1}{r}{romper} & 9\%   &  &  \\ 
2          & kuflowe                    & 7.2\% &  &  \\
3          & harnas                     & 6\%   &  &  \\
4          & ksiazece                   & 4.9   &  &  \\ 
\end{tabular}
\end{table}


 $(sin(x))^{2} +(cos(x))^{2} = 1$
 
 \begin{itemize}
    \item A
    \item G
    \item H
\end{itemize}

  random number:

\begin{enumerate}
    \item 2 
    \item 1
    \item 3
    \item 7
\end{enumerate}
 